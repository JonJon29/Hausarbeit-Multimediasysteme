%%%%%%%%%%%%%%%%%%%%%%%%%%%%%%%%%%%%%%%%%
% Journal Article
% LaTeX Template
% Version 1.4 (15/5/16)
%
% This template has been downloaded from:
% http://www.LaTeXTemplates.com
%
% Original author:
% Frits Wenneker (http://www.howtotex.com) with extensive modifications by
% Vel (vel@LaTeXTemplates.com)
% Adjustments by Dominik Rupprecht for Multimediasysteme
%
% License:
% CC BY-NC-SA 3.0 (http://creativecommons.org/licenses/by-nc-sa/3.0/)
%
%%%%%%%%%%%%%%%%%%%%%%%%%%%%%%%%%%%%%%%%%

%----------------------------------------------------------------------------------------
%	PACKAGES AND OTHER DOCUMENT CONFIGURATIONS
%----------------------------------------------------------------------------------------

\documentclass[11pt,a4paper]{article}

\bibliographystyle{alpha}

\usepackage[a4paper,margin=2.5cm]{geometry}

\usepackage{blindtext} % Package to generate dummy text throughout this template 
\usepackage{graphicx}
\usepackage[sc]{mathpazo} % Use the Palatino font
\usepackage[T1]{fontenc} % Use 8-bit encoding that has 256 glyphs
\linespread{1.05} % Line spacing - Palatino needs more space between lines
\usepackage{microtype} % Slightly tweak font spacing for aesthetics

\usepackage[ngerman]{babel} % Language hyphenation and typographical rules

\usepackage{lettrine} % The lettrine is the first enlarged letter at the beginning of the text

\usepackage{enumitem} % Customized lists
\setlist[itemize]{noitemsep} % Make itemize lists more compact

\usepackage{abstract} % Allows abstract customization
\renewcommand{\abstractnamefont}{\normalfont\bfseries} % Set the "Abstract" text to bold
\renewcommand{\abstracttextfont}{\normalfont\small\itshape} % Set the abstract itself to small italic text

\usepackage{titlesec} % Allows customization of titles
\titleformat{\section}[block]{\large\scshape}{\thesection.}{1em}{} % Change the look of the section titles
\titleformat{\subsection}[block]{\large}{\thesubsection.}{1em}{} % Change the look of the section titles

\usepackage{fancyhdr} % Headers and footers
\pagestyle{fancy} % All pages have headers and footers
\fancyhead{} % Blank out the default header
\fancyfoot{} % Blank out the default footer
\fancyhead[C]{Multimediasysteme $\bullet$ Januar/Februar 2026 } % Custom header text
\fancyfoot[CO]{\thepage} % Custom footer text

\usepackage{titling} % Customizing the title section
\usepackage{tabularx} % Tabellen mit automatischer Spaltenbreite

\usepackage{hyperref} % For hyperlinks in the PDF

\usepackage{algorithm} % Zur Darstellung von Pseudocode
\usepackage{algorithmic}
\floatname{algorithm}{Algorithmus} % Deutsche Beschriftung der Algorrithmen

%----------------------------------------------------------------------------------------
%	TITLE SECTION
%----------------------------------------------------------------------------------------

\setlength{\droptitle}{-4\baselineskip} % Move the title up

\pretitle{\begin{center}\Huge\bfseries} % Article title formatting
\posttitle{\end{center}} % Article title closing formatting
\title{Exposé: Bemerkbarkeit von Synchonisationsproblemen von Video und Audio} % Article title

\author{%
\large Eron Murseli \\ % Your name
\normalsize \href{mailto:eron.murseli1@informatik.hs-fulda.de}{eron.murseli1@informatik.hs-fulda.de}
\and
\large Jonas Möller \\ % Name des zweiten Autors
\normalsize \href{mailto:jonas-sven.moeller@informatik.hs-fulda.de}{jonas-sven.moeller@informatik.hs-fulda.de}
}
\date{\today} % Leave empty to omit a date

%----------------------------------------------------------------------------------------

\begin{document}

% Print the title
\maketitle


%----------------------------------------------------------------------------------------
%	ARTICLE CONTENTS
%----------------------------------------------------------------------------------------

\section{Ausgangslage und Problemstellung}
% Kontext und Motivation.
% Relevanz des Themas (wissenschaftlich, gesellschaftlich, praktisch).
% Klare Abgrenzung: Was gehört dazu, was nicht?

Filme bestehen sowohl aus Videomaterial als auch aus der dazugehörigen Audiospur.
Diese müssen für ein gutes Erlebnis synchronisiert sein.
Um abwägen zu können wie hoch die Toleranz dieser Diskrepanz sein darf, wollen wir herausfinden, wie bemerkbar Asychronität von Audio und Video ist.

\section{Forschungsfragen und Zielsetzung}
% Formuliere eine präzise, überprüfbare Forschungsfrage (oder 1–3 Teilfragen).
% Beschreibe die Zielsetzung (welche Erkenntnisse werden angestrebt?) und den erwarteten Neuheits- bzw. Mehrwert.

Wie hoch kann die Verzögerung zwischen Audio und Video bei einem Film sein, bis es von Konsumenten bemerkt wird?
Macht es einen Unterschied, ob die Audiospur zurückhängt, oder vorausläuft?
Ist die Bemerkbarkeit abhängig von dem Inhalt des Videos?

\section{Theoretischer Rahmen und Forschungsstand}
% ggf. Zentrale Begriffe und Theorien definieren.
% Kurzer Überblick über den relevanten Stand der Forschung (gezielt, nicht „alles“).
% Gib einen kurzen Übersicht, warum die Quelle wichtig für Dein Vorhaben ist.
% Forschungslücken benennen, die das Vorhaben adressiert.
% Füge hier nur einschlägige, bereits gesichtete Quellen ein.
% Beispiele: \cite{henning2007taschenbuch}, \cite{CiteDrive2022}.
% Es müssen mindestens drei wissenschaftliche Quellen zitiert werden, die Ihr gelesen habt und die zum Thema passen, z
%.B.:

% @TODO Forschungsstand fertigstellen

In einem Forschungsartikel von Kohlrausch, Armin and van de Par und Steven wurde bereits festgestellt, dass es
merkbarer ist, wenn die Audiospur dem Video vorausläuft. \cite{kohlrausch2000experimente}

% @TODO Papers berücksichtigen
\cite{grant2003discrimination}

\cite{lutzky2004guideline}

\section{Methodik und Daten}
% Gewählte Methode(n) begründen (z.B. Experiment, Befragung, Inhaltsanalyse, Leitfadeninterview).
% Untersuchungsdesign (Stichprobe, Rekrutierung, Materialien/Instrumente, Ablauf).
% Geplante Auswertung (qualitativ/quantitativ, Verfahren, Software).
% Qualitätssicherung (Validität/Reliabilität/Objektivität), Limitationen.

Die Daten sollen anhand von Befragungen erhoben werden.
Dabei werden den Probanden Videos mit Audio in unterschiedlichen Verzögerungen gezeigt.
Anschließend sollen diese angeben, ob es sich synchron angefühlt hat oder nicht.

Die Auswertung soll dann eine Tabelle sein, welche Zeigt ab welcher Verzögerung eine Asynchronität bemerkt wurde.
Diese soll nach Inhalt des Videos und nach positiver und negativer Verzögerung gegliedert werden.

% @TODO Qualitätssicherung festlegen

\section{Vorläufige Gliederung der Arbeit}
% Beispielhafte Struktur (anpassen oder detaillieren):

\begin{enumerate}[leftmargin=1cm,label=\arabic*.]
    \item Einleitung (Problem, Ziel, Aufbau)
    \item Theoretischer Rahmen und Forschungsstand
    \item Methodik
    \item Ergebnisse
    \item Diskussion (Einordnung, Limitationen, Implikationen)
    \item Fazit und Ausblick
\end{enumerate}

\section{Gruppenarbeit}
% kurze Beschreibung der Gruppenarbeit (z.B. Aufgabenverteilung, Kommunikation).
% Nur Bestandteil des Exposés: Für Abgabe der Hausarbeit nicht relevant.

Wir werden versuchen die Ausarbeitung gemeinsam zu verfassen.
Eine thematische Arbeitsteilung ist also nicht vorgesehen.
Die Zusammenarbeit soll über ein wöchentliches Treffen, vorzugsweise in Präsenz, stattfinden.

%----------------------------------------------------------------------------------------
%	Appendix
%----------------------------------------------------------------------------------------
\appendix
\section{Anhang}
% Fügen Sie hier ihre Anlagen an. Z.B. Angaben zur Nutzung von KI-Tools.


%----------------------------------------------------------------------------------------
%	REFERENCE LIST
%----------------------------------------------------------------------------------------
% \bibliographystyle{alpha}
\bibliography{literatur}

\end{document}
